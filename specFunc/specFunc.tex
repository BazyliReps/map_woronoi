\documentclass[hidelinks,10pt,a4paper]{article}
\usepackage[T1]{fontenc}
\usepackage[polish]{babel}
\usepackage[utf8]{inputenc}
\usepackage{lmodern}
\selectlanguage{polish}
\usepackage{amsmath}
\usepackage{amsfonts}
\usepackage{pdfpages}
\usepackage{hyperref}
\newcommand\tab[1][0.5cm]{\hspace*{#1}}


\begin{document}
\title{Algorytmy i Struktury Danych \\ projekt grupowy \\ specyfikacja funkcjonalna programu \\``Program''}
\author{Bazyli Reps}
\maketitle

\newpage

\tableofcontents

\newpage

\section{Wstęp}
\tab Celem projektu jest napisanie aplikacji umożliwiającej dzielenie obszaru na obszary na podstawie dostarczonych przez użytkownika współrzędnych miejsc symbolizujących instytucje finansowe, nazywane w dalszej części dokumentu \textbf{punktami kluczowymi}.
Aplikacja ta w dalszej części dokumentu będzie nazywana programem.

Poza punktami kluczowymi użytkownik dostarcza także informację dotyczącą \textbf{konturów}, czyli zbioru \textbf{współrzędnych} wyznaczających granicę symulowanego obszaru oraz \textbf{obiektów}, czyli określonych przez użytkownika bytów posiadających określone właściwości i znajdujących się na danym obszarze. Dane dostarczane są z pliku tekstowego zwanego dalej \textbf{plikiem wejściowym} lub z programu.
Dokładne informacje dotyczące podmiotów wypisanych pogrubioną czcionką znajdują się w sekcji \hyperref[sec:daneWej]{\textit{4 Format danych}}.

\section{Wymagania funkcjonalne}
\tab Program ma zawierać nastepujące funkcjonalności:
\begin{enumerate}


\item rysowanie i wczytywanie danych takich bytów jak:
\begin{enumerate}
\item \textbf{punkty kluczowe}
\item \textbf{kontury}
\item \textbf{obiekty}
\end{enumerate}
przy czym \textbf{punkty kluczowe} oraz \textbf{kontury} mają ściśle zdefiniowaną w tym dokumencie listę atrybutów, natomiast \textbf{obiekty} mają atrybuty definiowane przez użytkownika,
\item wyznaczanie i rysowanie granic na podstawie \textbf{konturów},
\item wyznaczenie i narysowanie optymalnych granic obszarów dookoła \textbf{punktów kluczowych}, nazywanych dalej \textbf{obszarami} przy czym:
\begin{enumerate}
\item w każdym z takich \textbf{obszarów} znajduje się wyłącznie jeden punkt kluczowy
\item optymalne granice mają wyznaczać takie \textbf{obszary}, w których każdy punkt należący do \textbf{obszaru} znajduje się bliżej \textbf{punktu kluczowego} tego \textbf{obszaru} niż jakiegokolwiek innego \textbf{punktu kluczowego}
\end{enumerate}
\item wyświetlanie posegregowanej według typów listy \textbf{obiektów} należących do każdego \textbf{obszaru}
\item tworzenie oraz wyświetlanie na żądanie użytkownika statystyk dotyczących danego obszaru, gdzie przez statystyki rozumie się:
\begin{enumerate}
\item podawanie liczby mieszkańców na podstawie informacji o \textbf{obiektach} zawierających atrybut typu \textit{L\_MIESZKANCOW} znajdujących się na \textbf{obszarze} 
\item wyznaczenie typów \textbf{obiektów} znajdujących się na \textbf{obszarze} oraz podanie ilości instancji \textbf{obiektów} dla każdego z typów
\end{enumerate}
\item podczas działania programu dodawanie i usuwanie elementów \textbf{konturów}
\item podczas działania programu dodawanie i usuwanie \textbf{punktów kluczowych} 
\item podkładanie pod rysowane \textbf{kontury} grafiki dostarczonej przez użytkownika

\end{enumerate}



\section{Wymagania niefunkcjonalne}
\tab Program ma pracować jako aplikacja internetowa, co oznacza, że będzie działał on na serwerze dostawcy usług a klient będzie komunikował się z nim przez przeglądarkę internetową za pomocą sieci WWW.
\subsection{obsługiwane przeglądarki}
Program ma być obsługiwany przez aktualne w czasie utworzenia dokumentu  przeglądarki internetowe takie jak:
\begin{itemize}
\item Mozilla Firefox
\item Google Chrome
\item Safari
\item Microsoft Edge
\end{itemize}  
\subsection{wykorzystywane technologie}
Program ma być napisany z wykorzystaniem następujących technologii [lista technologii]




\section{Format danych}
\label{sec:daneWej}

\subsection{pliki}
Użytkownik opcjonalnie dostarcza programowi następujące pliki:
\begin{enumerate}
\item Tekstowy plik o rozszerzeniu \textit{.txt} z enkodowaniem \textit{utf-8} zawierający informacje dotyczące \textbf{konturów}, \textbf{punktów kluczowych} oraz \textbf{obiektów} nazywany \textbf{plikiem wejściowym}
\item Grafikę w jednym z następujących formatów: 
\begin{enumerate}
\item .png
\item .jpg
\item .jpeg
\item .bmp
\item .gif
\end{enumerate}
która może służyć jako tło dla rysowanego układu nazywaną dalej   \textbf{grafiką}.
\end{enumerate}

\subsection{przykład poprawnie sformatowanego pliku wejściowego}
\# Kontury terenu (wymienione w kolejności łączenia): Lp. x y
\\1. 0 0
\\2. 0 20
\\3. 20 30.5
\\4. 40 20
\\5. 40 0
\\
\\
\# Punkty kluczowe: Lp. x y Nazwa
\\1. 1 1 KOK Krawczyka
\\2. 1 19 KOK Kaczmarskiego
\\3. 30 10 KOK Łazarewicza
\\
\\
\# Definicje obiektów: Lp. Typ\_obiektu (Nazwa\_zmiennej Typ\_zmiennej)
\\1. SZKOŁA Nazwa String X double Y double
\\2. DOM X double Y double L\_MIESZKAŃCÓW int
\\3. NIEDŹWIEDŹ X double Y double
\\
\\
\# Obiekty: Typ\_obiektu (zgodnie z definicją)
\\1. SZKOŁA "Szkoła robienia dużych pieniędzy" 4 1
\\2. DOM 4 3 100
\\3. DOM 4 17 120
\\4. DOM 4 18 80
\\5. NIEDŹWIEDŹ 20 20
\\6. NIEDŹWIEDŹ 40 1
\\7. NIEDŹWIEDŹ 39 1
\\8. NIEDŹWIEDŹ 39 2
\\
\subsection{opis typów}


\subsubsection{kontury / współrzędna}
\textbf{Kontury} stanowią listę obiektów typu \textbf{współrzędna}.
Kolejność w liście odwzorowuje kolejność dodawania \textbf{współrzędnych} do listy w ten sposób, że pierwsza dodana do \textbf{konturów} \textbf{współrzędna} będzie pierwszą \textbf{współrzędną} na liście. Każda z \textbf{współrzędnych} zawiera dokładnie dwa atrybuty:
\begin{enumerate}
\item \textit{x} - nieujemna liczba całkowita oznaczająca wartość odciętych w dwuwymiarowym układzie współrzędnych
\item \textit{y} - nieujemna liczba całkowita oznaczająca wartość rzędnych w dwuwymiarowym układzie współrzędnych
\end{enumerate}

\subsubsection{punkt kluczowy}
\textbf{Punkt kluczowy} symbolizuje instytucję finansową. 
Zawiera on dokładnie trzy atrybuty:
\begin{enumerate}
\item \textit{x} - nieujemna liczba całkowita oznaczająca wartość odciętych w dwuwymiarowym układzie współrzędnych
\item \textit{y} - nieujemna liczba całkowita oznaczająca wartość rzędnych w dwuwymiarowym układzie współrzędnych
\item \textit{Nazwa} - nazwa instytucji. Całość nazwy razem ze spacjami lub innymi białymi znakami nie może przekraczać 100 znaków. 
\end{enumerate}

\subsubsection{obiekt}
\textbf{Obiekt} jest definiowany przez użytkownika w \textbf{pliku wejściowym} lub bezpośrednio w programie. 
\textbf{Obiekt} może zawierać od dwóch do sześciu atrybutów pośród których koniecznie muszą znaleźć się dwa atrybuty:
\begin{enumerate}
\item \textit{x} - nieujemna liczba całkowita oznaczająca wartość odciętych w dwuwymiarowym układzie współrzędnych
\item \textit{y} - nieujemna liczba całkowita oznaczająca wartość rzędnych w dwuwymiarowym układzie współrzędnych
\end{enumerate}
Atrybuty w \textbf{obiekcie} mogą być dodawane w dowolnej kolejności.

\subsection{opis formatowania}

\textbf{Plik wejściowy} musi składać się z dokładnie czterech sekcji, z których każda z nich musi zaczynać się linią w której pierwszym znakiem jest znak '\#' zwaną dalej \textbf{linią rozpoczynającą}.
W dalszej części \textbf{linii rozpoczynającej} użytkownik może wpisać dowolne znaki.
Każda z linii nie może zawierać więcej niż 200 znaków.
Elementy każdej sekcji muszą znajdować się w kolejnych liniach. 
Elementy poza \textbf{linią rozpoczynającą} w każdej sekcji muszą być numerowane kolejnymi liczbami naturalnymi (rozpoczynając od liczby '1') po których bezpośrednio występuje znak kropki '.' i co najmniej jedna spacja.
 

\subsubsection{kontury}





\section{Sposób uruchomienia programu}








\section{Sytuacje wyjątkowe}

\subsection{Błędy pliku}


\subsection{Błędy formatowania pliku}




\section{Testy akceptacyjne}





\end{document}
