\documentclass[a4paper, 10pt, titlepage]{article}

\usepackage{polski}
\usepackage{indentfirst}
\usepackage[utf8]{inputenc}

\newcommand{\code}[1]{\texttt{#1}}

\usepackage{fancyhdr}
\pagestyle{fancy} 

\title {Projekt grupowy [GR3] -- 2018/2019\\Specyfikacja implementacyjna}
\author{Grupa 7 \\\\ Sebastian Stasiak, indeks 291109 \\ Bazyli Reps, indeks ******}

\begin{document}

\maketitle
\tableofcontents
\newpage

\section{Wykorzystane technologie}

Projekt napisany jest w postaci aplikacji web'owej.
Po stronie serwera wykorzystany jest framework \code{ASP.NET MVC 5}. Kod po stronie serwera pisany jest w \code{.NET Framework 4.7.1}, albo w \code{.NET Standard} jeśli nie ma potrzeby korzystania z żadnych funkcjonalności specyficznych dla środowiska Windows.
Warstwa prezentacji opierać będzie się na plikach Razor (\code{.cshtml}).

Po stronie użytkownika wykorzystany zostaje \code{JavaScript} (z założeniem, że przeglądarka użytkownika obsługuje \code{ECMAScript 9}), w tym popularna biblioteka \code{JQuery} oraz szereg pomniejszych bibliotek.
\code{CSS} ujednolicony będzie poprzez wykorzystanie biblioteki \code{Bootstrap 4}.

\section{Serwer}

Aplikacja działać będzie w chmurze, przy wykorzystaniu platformy Azure.
Możliwe jest także jej postawienie na dowolnej maszynie na której znajduje się system Windows 10 bądź Windows Server 2016.
W tym celu należy zainstalować na niej \code{.NET Framework Runtime 4.7.1} bądź nowszy, oraz IIS 10.
Działanie na starszych wersjach jest możliwe, jednak nie jest gwarantowane i nie będzie testowane.

\end{document}